% Exemple de CV utilisant la classe moderncv
% Style classic en bleu
% Article complet : http://blog.madrzejewski.com/creer-cv-elegant-latex-moderncv/

\documentclass[10pt,a4paper]{moderncv}
\moderncvtheme[blue]{classic}                
\usepackage[utf8]{inputenc}
\usepackage[top=1.1cm, bottom=1.1cm, left=1.8cm, right=1.8cm]{geometry}
% Largeur de la colonne pour les dates
\setlength{\hintscolumnwidth}{2.5cm}

\firstname{Philippe}
\familyname{Inthavong}
\title{Ing\'{e}nieur Java JEE}              
\address{21 rue Jean Baptiste Cl\'{e}ment}{94800 Villejuif}    
\email{philippe.inthavong@gmail.com}                      
%\homepage{www.madrzejewski.com}
\mobile{06 89 95 84 68} 
\extrainfo{26 ans -- Permis B}
\begin{document}
\maketitle

\section{Exp\'{e}riences}

\cventry{Septembre 2015\\à Aujourd'hui}{D\'eveloppeur Java JEE}{Sopra Steria (client: Caisse nationale de l'assurance maladie des travailleurs salari\'{e}s)}{}{Paris}{
\begin{itemize}
	\item Rédaction de spécifications techniques
	\item Conception applicative
	\item Développement et tests
	\item Rédaction de documents techniques
	\item Encadrement technique
	\end{itemize}}

	\cventry{Juin 2015\\à Septembre 2015}{D\'eveloppeur}{Sopra Steria (client: Assistance Publique des Hôpitaux de Paris)}{}{Paris}{Maintenance corrective et \'{e}volutive de l'outil de gestion du r\'{e}f\'{e}rentiel des structures de l'AP-HP.
	\begin{itemize}
	\item Conception et développement de lots évolutifs
	\item Organisation de la réversibilité du projet pour les équipes clientes
	\end{itemize}}

	\cventry{Mars 2014\\à Septembre 2014}{Ingénieur R\&D }{Laboratoire d’ing\'{e}nierie du Mouvement}{Montr\'{e}al}{\newline{} \textit{Développement d'un outil opto-electronique anthropométrique}}
	{Développement d'un outil de mesure à bas coût basé sur les caméras Kinect pour la prise de mesures anthropométriques (volume et masse) du corps humain.\newline{}}

	\cventry{Juin 2013\\à Ao\^ut 2013}{Etudiant-Chercheur}{Laboratoire d’ing\'{e}nierie du Mouvement}{Montr\'{e}al}
	{\newline{} \textit{Développement d'un mod\`ele tridimensionnel de la coiffes des rotateurs}}
	{Développement sous Matlab d'un mod\`ele musculo-squelettique de l'\'{e}paule. \newline{}}

	\section{Formations}
	\cventry{2011 -- 2014}{Diplôme d'ing\'{e}nieur}{Ecole T\'{e}l\'{e}com Physique Strasbourg}{}{Sp\'{e}cialit\'{e} Ing\'{e}nierie et Science pour le Vivant}{}
	\cventry{}{Master IRIV}{Universit\'{e} de Strasbourg}{}{Imagerie, Robotique et Ing\'{e}nierie pour le Vivant}{}
	\cventry{2009 -- 2011}{Classe pr\'{e}paratoire aux grandes \'{e}coles}{Lyc\'{e}e Paul Val\'{e}ry (Paris)}{}{Sp\'{e}cialit\'{e} Math\'{e}matiques, Physiques}{}
	\cventry{2009}{Etudes secondaires}{Lyc\'{e}e EPIN (94)}{}{Baccalaur\'{e}at Scientifique, mention Bien}{}

	\section{Comp\'{e}tences en informatique}
	\cvitem{Langages}{Java, Javascript, HTML, CSS, SQL, Bash, C/C++, Python}
	\cvitem{Base de donn\'{e}es}{Oracle, ProgresSQL, MySQL, MongoDB}
	\cvitem{Serveurs}{Apache Tomcat, JBoss Application Server, Oracle Weblogic Server 11g}
	\cvitem{Systèmes}{Windows 7, MacOS X, Linux (Ubuntu)}
	\cvitem{Outils}{Eclipse, Ant, Maven, Jenkins, Sonar, Phabricator, Microsoft office}

	\section{Langues}
	\cvlanguage{Anglais}{lu, \'{e}crit, parl\'{e} -- Toeic 900}{}
	\cvlanguage{Allemand}{Scolaire}{}
	\cvlanguage{Japonais}{Débutant}{}

	\section{Centres d'int\'{e}r\^et}
	\cvitem{Culture}{Photographie, Cin\'{e}ma (science-fiction), Voyages}
	\cvitem{Sport}{Basket Ball (loisir), Taekwondo (2 ann\'{e}es en club)}  
	\cvitem{Association}{Pr\'{e}sident (2 ann\'{e}es) du « Club Photo » de T\'{e}l\'{e}com Physique Strasbourg \newline
	Membre (1 ann\'{e}e) du « Bureau des f\^etes » de T\'{e}l\'{e}com Physique Strasbourg}
\end{document}

